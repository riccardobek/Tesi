% !TEX encoding = UTF-8
% !TEX TS-program = pdflatex
% !TEX root = ../tesi.tex

%**************************************************************
% Sommario
%**************************************************************
\cleardoublepage
\phantomsection
\pdfbookmark{Sommario}{Sommario}
\begingroup
\let\clearpage\relax
\let\cleardoublepage\relax
\let\cleardoublepage\relax

\chapter*{Sommario}

Il presente documento descrive il lavoro svolto durante il periodo di stage, della durata di circa trecentodieci ore, dal laureando \myName\ presso l'azienda \azienda .
Gli obiettivi da raggiungere erano molteplici.\\
In primo luogo era richiesto uno studio e apprendimento degli strumenti adottati dalla azienda per lo sviluppo delle applicazioni e dei data warehouse che sono rispettivamente \inde\ e  Microsoft SQL Server. In secondo luogo era richiesto di seguire una buona metodologia nello sviluppo delle unità della componente da realizzare effettuando molteplici test. Inoltre, al termine dello stage, in caso di raggiungimento dello stato di validazione e collaudo, si era prefissato il rilascio del progetto con eventuale gestione autonoma di modifiche correttive e/o adattive segnalate dal cliente.\\
Gli obiettivi in generale erano mirati all'apprendimento di come \azienda\ gestisce i suoi clienti e realizza i software richiesti. Inoltre, avevano l'ulteriore obiettivo di andare a migliorare le capacità acquisite nella disciplina di Ingegneria del Software.\\
\\
Questa tesi si compone di \todo capitoli.
% Nel primo viene delineato il profilo dell'azienda e le metodologie di lavoro della stessa. Il secondo capitolo, invece, presenta (anche a livello tecnologico) il progetto al centro delle attività svolte durante lo stage, che verranno approfondite (suddivise in base agli obiettivi) nel terzo capitolo. Infine, il quarto capitolo presenta una valutazione retrospettiva del tirocinio, sia a livello oggettivo, considerando, ad esempio, il grado di soddisfacimento degli obiettivi, che soggettivo, esponendo, quindi, una mia valutazione personale su quanto svolto.



%della componente da andare a sviluppare, effettuando incontri con il cliente e collaborando all'individuazione di una soluzione adeguata al problema.
%In secondo luogo era richiesto lo sviluppo dei diagrammi Entity Relationship relativi ai database necessari all'implementazione del progetto con un eventuale bozza di quello che sarebbe stato il risultato finale del progetto. 
%In secondo luogo era richiesta l'implementazione della componente. 
%Tale prodotto permette di mostrare il catalogo di tutti i prodotti dell'azienda cliente e di gestire le informazioni relative ai singoli prodotti (immagini principali, descrizione breve, descrizione estesa, nome esteso del prodotto, quantità, ecc.).
%Terzo ed ultimo obbiettivo era, dopo aver verificato, validato e collaudato internamente il prodotto, l'integrazione della componente al software già esistente, restando in attesa di eventuali modifiche correttive e/o adattive da parte del cliente.

%\vfill
%
%\selectlanguage{english}
%\pdfbookmark{Abstract}{Abstract}
%\chapter*{Abstract}
%
%\selectlanguage{italian}

\endgroup			

\vfill

