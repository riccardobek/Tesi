% !TEX encoding = UTF-8
% !TEX TS-program = pdflatex
% !TEX root = ../tesi.tex

%**************************************************************
% Sommario
%**************************************************************
\cleardoublepage
\phantomsection
\pdfbookmark{Sommario}{Sommario}
\begingroup
\let\clearpage\relax
\let\cleardoublepage\relax
\let\cleardoublepage\relax

\chapter*{Sommario}

Il presente documento descrive il lavoro svolto durante il periodo di stage, della durata di circa trecentodieci ore, dal laureando \myName\ presso l'azienda \azienda .
Gli obiettivi da raggiungere sono stati molteplici.\\
In primo luogo hanno richiesto uno studio e apprendimento degli strumenti adottati dall'azienda per lo sviluppo delle applicazioni e dei data warehouse che sono rispettivamente \inde\ e  Microsoft SQL Server. In secondo luogo ho dovuto seguire una buona metodologia nello sviluppo delle unità della componente da realizzare effettuando molteplici test. Inoltre, al termine dello stage, in caso di raggiungimento dello stato di validazione e collaudo, si è prefissato il rilascio del progetto con eventuale gestione autonoma di modifiche correttive e/o adattive segnalate dal cliente.\\
Gli obiettivi in generale sono mirati all'apprendimento di come \azienda\ gestisce i suoi clienti e realizza i software richiesti. In aggiunta, sono serviti a perseguire l'ulteriore obiettivo di andare a migliorare le capacità acquisite nella disciplina di Ingegneria del Software.\\
\\
Questa tesi si compone di 4 capitoli. 
Il primo presenta l'azienda, come è nata, quali tecnologie e quale metodologia di lavoro adotta. Il secondo, invece, presenta il progetto al centro delle attività svolte durante lo stage, i vincoli e gli obiettivi prefissati. Nel terzo capitolo viene presentato il progetto nel dettaglio presentando le scelte di progettazione e implementazione seguite da una piccola digressione su \inde\.
Infine, il quarto capitolo presenta una valutazione del tirocinio, sia a livello oggettivo, considerando, ad esempio, il grado di soddisfacimento degli obiettivi, che soggettivo, esponendo, quindi, una mia valutazione personale sulle attività svolte.


%\vfill
%
%\selectlanguage{english}
%\pdfbookmark{Abstract}{Abstract}
%\chapter*{Abstract}
%
%\selectlanguage{italian}

\endgroup			

\vfill

