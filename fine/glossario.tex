\chapter{Glossario}

\paragraph{Accordo di non divulgazione}
\label{NDA}
Un accordo di non divulgazione (in lingua inglese Non-Disclosure Agreement, NDA) è un negozio giuridico di natura sinallagmatica che designa informazioni confidenziali e con il quale le parti si impegnano a mantenerle segrete, pena la violazione dell'accordo stesso e il decorso di specifiche clausole penali in esso contenute. \hyperref[bib13]{\cite{[13]}}

\paragraph{CRUD}
\label{CRUD}
Il termine CRUD è l'abbreviazione dei termini inglesi create, read, update e delete. Queste rappresentano le quattro funzioni di base dei database relazionali.

\paragraph{IDE}
\label{IDE}
Un ambiente di sviluppo integrato (in lingua inglese Integrated Development Environment), è un software che, in fase di programmazione, aiuta i programmatori nello sviluppo del codice sorgente di un programma. 
Spesso l'IDE aiuta lo sviluppatore segnalando errori di sintassi del codice direttamente in fase di scrittura, oltre a tutta una serie di strumenti e funzionalità di supporto alla fase di sviluppo e debugging. \hyperref[bib11]{\cite{[16]}}

\paragraph{Instant Developer}
\label{InDe}
Instant Developer, in particolare la versione per desktop, Foundation è la piattaforma di sviluppo adottata per creare il progetto oggetto della tesi. Si tratta di un RAD che permette di realizzare applicazioni in tempi molto brevi senza la necessità di conoscere il codice alla base del progetto.

\paragraph{JDBC}
\label{JDBC}
L'API JDBC (Java Database Connectivity) è lo standard industriale per la connettività dei database tra il linguaggio di programmazione Java e un'ampia gamma di database: database SQL e altre origini dati tabulari, come fogli di calcolo o altri file. L'API JDBC fornisce un'API a livello di chiamata per l'accesso al database basato su SQL \cite{[26]}.

\paragraph{ODBC}
\label{ODBC}
In informatica Open DataBase Connectivity (ODBC) è una API standard per la connessione dal client al DBMS. Questa API è indipendente dai linguaggi di programmazione, dai sistemi di database e dal sistema operativo \cite{[27]}.

\paragraph{UML} 
\label{UMLl} 
In ingegneria del software \emph{UML, Unified Modeling Language} (ing. linguaggio di modellazione unificato) è un linguaggio di modellazione e specifica basato sul paradigma object-oriented. L'\emph{UML} svolge un'importantissima funzione di ``lingua franca'' nella comunità della progettazione e programmazione a oggetti. Gran parte della letteratura di settore usa tale linguaggio per descrivere soluzioni analitiche e progettuali in modo sintetico e comprensibile a un vasto pubblico


