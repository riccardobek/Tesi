\chapter{Glossario}
%\paragraph{API} 
%\label{API} 
%In informatica con il termine \emph{Application Programming Interface API} (ing. interfaccia di programmazione di un'applicazione) si indica ogni insieme di procedure disponibili al programmatore, di solito raggruppate a formare un set di strumenti specifici per l'espletamento di un determinato compito all'interno di un certo programma. La finalità è ottenere un'astrazione, di solito tra l'hardware e il programmatore o tra software a basso e quello ad alto livello semplificando così il lavoro di programmazione

\paragraph{UML} 
\label{UMLl} 
In ingegneria del software \emph{UML, Unified Modeling Language} (ing. linguaggio di modellazione unificato) è un linguaggio di modellazione e specifica basato sul paradigma object-oriented. L'\emph{UML} svolge un'importantissima funzione di ``lingua franca'' nella comunità della progettazione e programmazione a oggetti. Gran parte della letteratura di settore usa tale linguaggio per descrivere soluzioni analitiche e progettuali in modo sintetico e comprensibile a un vasto pubblico

%\paragraph{C\#}\label{C#}
%Linguaggio di programmazione orientato agli oggetti che consente di creare una vasta gamma di applicazioni protette e affidabili per .NET Framework. Esso può essere adottato per creare applicazioni client Windows, servizi Web XML, componenti distribuiti, applicazioni client\-server, applicazioni di database e molto altro.
%
%\paragraph{Java}
%Linguaggio di programmazione ad alto livello, orientato agli oggetti e a tipizzazione statica, che si appoggia sull'omonima piattaforma software, specificamente progettato per essere il più possibile indipendente dalla piattaforma hardware di esecuzione. \cite{[4]}

\paragraph{IDE}
\label{IDE}
Un ambiente di sviluppo integrato (in lingua inglese Integrated Development Environment), è un software che, in fase di programmazione, aiuta i programmatori nello sviluppo del codice sorgente di un programma. 
Spesso l'IDE aiuta lo sviluppatore segnalando errori di sintassi del codice direttamente in fase di scrittura, oltre a tutta una serie di strumenti e funzionalità di supporto alla fase di sviluppo e debugging. \cite{[16]}

\paragraph{Accordo di non divulgazione}
\label{NDA}
Un accordo di non divulgazione (in lingua inglese Non-Disclosure Agreement, NDA) è un negozio giuridico di natura sinallagmatica che designa informazioni confidenziali e con il quale le parti si impegnano a mantenerle segrete, pena la violazione dell'accordo stesso e il decorso di specifiche clausole penali in esso contenute. \cite{[13]}

\paragraph{Instant Developer}
\label{InDe}
Instant Developer, in particolare la versione per desktop, Foundation è la piattaforma di sviluppo adottata per creare il progetto oggetto della tesi. Si tratta di un RAD che permette di realizzare applicazioni in tempi molto brevi senza la necessità di conoscere il codice alla base del progetto.

\paragraph{CRUD}
\label{CRUD}
Il termine CRUD è l'abbreviazione dei termini inglesi create, read, update e delete. Queste rappresentano le quattro funzioni di base dei database relazionali.
