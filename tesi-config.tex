%**************************************************************
% file contenente le impostazioni della tesi
%**************************************************************

%**************************************************************
% Frontespizio
%**************************************************************

% Autore
\newcommand{\myName}{Riccardo Bernucci}                                    
\newcommand{\myTitle}{Analisi, Progettazione e Implementazione di un componente Software adottando Instant Developer}

% Tipo di tesi                   
\newcommand{\myDegree}{Tesi di laurea triennale}

% Università             
\newcommand{\myUni}{Università degli Studi di Padova}

% Facoltà       
\newcommand{\myFaculty}{Corso di Laurea in Informatica}

% Dipartimento
\newcommand{\myDepartment}{Dipartimento di Matematica "Tullio Levi-Civita"}

% Titolo del relatore
\newcommand{\profTitle}{Prof.}

% Relatore
\newcommand{\myProf}{Francesco Ranzato}

% Luogo
\newcommand{\myLocation}{Padova}

% Anno accademico
\newcommand{\myAA}{2018-2019}

% Data discussione
\newcommand{\myTime}{Luglio 2019}


%**************************************************************
% Impostazioni di impaginazione
% see: http://wwwcdf.pd.infn.it/AppuntiLinux/a2547.htm
%**************************************************************

\setlength{\parindent}{14pt}   % larghezza rientro della prima riga
\setlength{\parskip}{0pt}   % distanza tra i paragrafi


%**************************************************************
% Impostazioni di biblatex
%**************************************************************
\bibliography{bibliografia} % database di biblatex 

\defbibheading{bibliography} {
    \cleardoublepage
    \phantomsection 
    \addcontentsline{toc}{chapter}{\bibname}
    \chapter*{\bibname\markboth{\bibname}{\bibname}}
}

\setlength\bibitemsep{1.5\itemsep} % spazio tra entry

\DeclareBibliographyCategory{opere}
\DeclareBibliographyCategory{web}

\addtocategory{opere}{womak:lean-thinking}
\addtocategory{web}{site:agile-manifesto}

\defbibheading{opere}{\section*{Riferimenti bibliografici}}
\defbibheading{web}{\section*{Siti Web consultati}}


%**************************************************************
% Impostazioni di caption
%**************************************************************
\captionsetup{
    tableposition=top,
    figureposition=bottom,
    font=small,
    format=hang,
    labelfont=bf
}

%**************************************************************
% Impostazioni di glossaries
%**************************************************************
\chapter{Glossario}
\paragraph{API} 
\label{API} 
In informatica con il termine \emph{Application Programming Interface API} (ing. interfaccia di programmazione di un'applicazione) si indica ogni insieme di procedure disponibili al programmatore, di solito raggruppate a formare un set di strumenti specifici per l'espletamento di un determinato compito all'interno di un certo programma. La finalità è ottenere un'astrazione, di solito tra l'hardware e il programmatore o tra software a basso e quello ad alto livello semplificando così il lavoro di programmazione

\paragraph{UML} 
\label{UML} 
In ingegneria del software \emph{UML, Unified Modeling Language} (ing. linguaggio di modellazione unificato) è un linguaggio di modellazione e specifica basato sul paradigma object-oriented. L'\emph{UML} svolge un'importantissima funzione di ``lingua franca'' nella comunità della progettazione e programmazione a oggetti. Gran parte della letteratura di settore usa tale linguaggio per descrivere soluzioni analitiche e progettuali in modo sintetico e comprensibile a un vasto pubblico

\paragraph{C\#}
Linguaggio di programmazione orientato agli oggetti che consente di creare una vasta gamma di applicazioni protette e affidabili per .NET Framework. Esso può essere adottato per creare applicazioni client Windows, servizi Web XML, componenti distribuiti, applicazioni client\-server, applicazioni di database e molto altro.

\paragraph{Java}
Linguaggio di programmazione ad alto livello, orientato agli oggetti e a tipizzazione statica, che si appoggia sull'omonima piattaforma software, specificamente progettato per essere il più possibile indipendente dalla piattaforma hardware di esecuzione.

\paragraph{IDE}
\label{IDE}
Un ambiente di sviluppo integrato (in lingua inglese Integrated Development Environment), è un software che, in fase di programmazione, aiuta i programmatori nello sviluppo del codice sorgente di un programma. 
Spesso l'IDE aiuta lo sviluppatore segnalando errori di sintassi del codice direttamente in fase di scrittura, oltre a tutta una serie di strumenti e funzionalità di supporto alla fase di sviluppo e debugging.

\paragraph{Accordo di non divulgazione}
\label{NDA}
Un accordo di non divulgazione (in lingua inglese Non-Disclosure Agreement, NDA) è un negozio giuridico di natura sinallagmatica che designa informazioni confidenziali e con il quale le parti si impegnano a mantenerle segrete, pena la violazione dell'accordo stesso e il decorso di specifiche clausole penali in esso contenute.

\paragraph{Instant Developer}
\label{InDe}
Instant Developer, in particolare la versione per desktop, Foundation è la piattaforma di sviluppo adottata per creare il progetto oggetto della tesi. Si tratta di un RAD che permette di realizzare applicazioni in tempi molto brevi senza la necessità di conoscere il codice alla base del progetto. 
 % database di termini
\makeglossaries


%**************************************************************
% Impostazioni di graphicx
%**************************************************************
\graphicspath{{immagini/}} % cartella dove sono riposte le immagini


%**************************************************************
% Impostazioni di hyperref
%**************************************************************
\hypersetup{
    %hyperfootnotes=false,
    %pdfpagelabels,
    %draft,	% = elimina tutti i link (utile per stampe in bianco e nero)
    colorlinks=true,
    linktocpage=true,
    pdfstartpage=1,
    pdfstartview=FitV,
    % decommenta la riga seguente per avere link in nero (per esempio per la stampa in bianco e nero)
    %colorlinks=false, linktocpage=false, pdfborder={0 0 0}, pdfstartpage=1, pdfstartview=FitV,
    breaklinks=true,
    pdfpagemode=UseNone,
    pageanchor=true,
    pdfpagemode=UseOutlines,
    plainpages=false,
    bookmarksnumbered,
    bookmarksopen=true,
    bookmarksopenlevel=1,
    hypertexnames=true,
    pdfhighlight=/O,
    %nesting=true,
    %frenchlinks,
    urlcolor=webbrown,
    linkcolor=RoyalBlue,
    citecolor=webgreen,
    %pagecolor=RoyalBlue,
    %urlcolor=Black, linkcolor=Black, citecolor=Black, %pagecolor=Black,
    pdftitle={\myTitle},
    pdfauthor={\textcopyright\ \myName, \myUni, \myFaculty},
    pdfsubject={},
    pdfkeywords={},
    pdfcreator={pdfLaTeX},
    pdfproducer={LaTeX}
}

%**************************************************************
% Impostazioni di itemize
%**************************************************************
\renewcommand{\labelitemi}{$\ast$}

%\renewcommand{\labelitemi}{$\bullet$}
%\renewcommand{\labelitemii}{$\cdot$}
%\renewcommand{\labelitemiii}{$\diamond$}
%\renewcommand{\labelitemiv}{$\ast$}


%**************************************************************
% Impostazioni di listings
%**************************************************************
\lstset{
    language=[LaTeX]Tex,%C++,
    keywordstyle=\color{RoyalBlue}, %\bfseries,
    basicstyle=\small\ttfamily,
    %identifierstyle=\color{NavyBlue},
    commentstyle=\color{Green}\ttfamily,
    stringstyle=\rmfamily,
    numbers=none, %left,%
    numberstyle=\scriptsize, %\tiny
    stepnumber=5,
    numbersep=8pt,
    showstringspaces=false,
    breaklines=true,
    frameround=ftff,
    frame=single
} 


%**************************************************************
% Impostazioni di xcolor
%**************************************************************
\definecolor{webgreen}{rgb}{0,.5,0}
\definecolor{webbrown}{rgb}{.6,0,0}


%**************************************************************
% Altro
%**************************************************************

\newcommand{\omissis}{[\dots\negthinspace]} % produce [...]

% eccezioni all'algoritmo di sillabazione
\hyphenation
{
    ma-cro-istru-zio-ne
    gi-ral-din
}

\newcommand{\sectionname}{sezione}
\addto\captionsitalian{\renewcommand{\figurename}{Figura}
                       \renewcommand{\tablename}{Tabella}}

\newcommand{\glsfirstoccur}{\ap{{[g]}}}

\newcommand{\intro}[1]{\emph{\textsf{#1}}}

%**************************************************************
% Environment per ``rischi''
%**************************************************************
\newcounter{riskcounter}                % define a counter
\setcounter{riskcounter}{0}             % set the counter to some initial value

%%%% Parameters
% #1: Title
\newenvironment{risk}[1]{
    \refstepcounter{riskcounter}        % increment counter
    \par \noindent                      % start new paragraph
    \textbf{\arabic{riskcounter}. #1}   % display the title before the 
                                        % content of the environment is displayed 
}{
    \par\medskip
}

\newcommand{\riskname}{Rischio}

\newcommand{\riskdescription}[1]{\textbf{\\Descrizione:} #1.}

\newcommand{\risksolution}[1]{\textbf{\\Soluzione:} #1.}

%**************************************************************
% Environment per ``use case''
%**************************************************************
\newcounter{usecasecounter}             % define a counter
\setcounter{usecasecounter}{0}          % set the counter to some initial value

%%%% Parameters
% #1: ID
% #2: Nome
\newenvironment{usecase}[2]{
    \renewcommand{\theusecasecounter}{\usecasename #1}  % this is where the display of 
                                                        % the counter is overwritten/modified
    \refstepcounter{usecasecounter}             % increment counter
    \vspace{10pt}
    \par \noindent                              % start new paragraph
    {\large \textbf{\usecasename #1: #2}}       % display the title before the 
                                                % content of the environment is displayed 
    \medskip
}{
    \medskip
}

\newcommand{\usecasename}{UC}

\newcommand{\usecaseactors}[1]{\textbf{\\Attori Principali:} #1. \vspace{4pt}}
\newcommand{\usecasepre}[1]{\textbf{\\Precondizioni:} #1. \vspace{4pt}}
\newcommand{\usecasedesc}[1]{\textbf{\\Descrizione:} #1. \vspace{4pt}}
\newcommand{\usecasepost}[1]{\textbf{\\Postcondizioni:} #1. \vspace{4pt}}
\newcommand{\usecasealt}[1]{\textbf{\\Scenario Alternativo:} #1. \vspace{4pt}}

%**************************************************************
% Environment per ``namespace description''
%**************************************************************

\newenvironment{namespacedesc}{
    \vspace{10pt}
    \par \noindent                              % start new paragraph
    \begin{description} 
}{
    \end{description}
    \medskip
}

\newcommand{\classdesc}[2]{\item[\textbf{#1:}] #2}

%**************************************************************
% Nuovi comandi
%**************************************************************
\newcommand{\azienda}{Tepui S.r.l}
\newcommand{\inde}{Instant Developer}
\newcommand{\todo}{\colorbox{red}{TO DO}}