% !TEX encoding = UTF-8
% !TEX TS-program = pdflatex
% !TEX root = ../tesi.tex


\chapter{L'Azienda}
\label{cap1:L'Azienda}
%**************************************************************

%Introduzione al contesto applicativo.\\

%\noindent Esempio di utilizzo di un termine nel glossario \\
%\gls{api}. \\

%\noindent Esempio di citazione in linea \\
%\cite{site:agile-manifesto}. \\

%\noindent Esempio di citazione nel pie' di pagina \\
%citazione\footcite{womak:lean-thinking} \\

%**************************************************************

\textit{Questo capitolo descrive nel dettaglio l'azienda ospitante andando a definire il suo business, l’organizzazione interna e le tecnologie adottate per soddisfare i clienti.}

\section{Presentazione dell'azienda}
\label{cap1:Presentazione dell'azienda}
Questa sezione si concentra sull'azienda che ha ospitato lo stage, fornendo una chiara spiegazione dei prodotti e dei servizi forniti assieme alla descrizione delle tecnologie adottate.

\subsection{\azienda}
\label{cap1:Tepui}
\paragraph*{}\azienda\ è una software house specializzata nello sviluppo di applicazioni gestionali attraverso l'utilizzo di strumenti CASE. 

\paragraph*{}Nasce nel 2016 dall'idea di tre persone. I fondatori hanno incontrato diverse realtà lavorative prima di portare avanti il progetto di aprire un'azienda. Due di queste hanno lavorato nell'ambito della business intelligence. Sulla base degli studi effettuati, risulta molto difficile analizzare i dati delle imprese in quanto ognuna ha un sua metodologia lavorativa, soprattutto nella gestione dei database o data warehouse. Sono arrivati alla conclusione che, se fosse possibile fornire un prodotto software standardizzato, sarebbe molto più semplice anche l'analisi. In ragione di ciò, i due hanno iniziato a cercare un'applicazione che permettesse di creare software in maniera rapida e compatibile con diversi sistemi operativi. Il risultato della loro ricerca è stato \inde\ (\hyperref[InDe]{InDe}) ed a partire da questo momento si è unito il terzo fondatore già esperto dell'applicazione che li ha formati nel suo utilizzo.

\begin{figure}[!h] 
	\centering 
	\includegraphics[width=0.5\columnwidth]{logo-tepui} 
	\caption{Logo dell'azienda}
\end{figure}

 
\paragraph*{}I tre fondatori sono fortemente convinti nei vantaggi tecnologici ed economici che l'implementazione mediante strumenti di sviluppo rapido porta ai propri clienti. 
L'azienda nasce come evoluzione di tre progetti indipendenti dei tre soci fondatori nelle aree della consulenza direzionale, consulenza nei processi produttivi, business intelligence  e produzione di software. 
Inizialmente, il software è stato utilizzato come strumento a supporto della consulenza e, l'esigenza di sviluppare velocemente applicazioni database driven, ha portato alla ricerca di una soluzione rapida ed efficiente che fosse focalizzata sui processi e non sulla programmazione. 
La potenza e versatilità della soluzione adottata ha portato a fornire anche esternamente servizi di sviluppo e software basati su tale tecnologia. 
Contestualmente a questo sviluppo, l'azienda si è specializzata nella realizzazione di soluzione di business intelligence attraverso realtà aziendali separate, ma fortemente interconnesse da un punto di vista della visione imprenditoriale.
Tepui si è affermata quindi come azienda che concentra queste esperienze e le propone come fornitore unico di consulenza e prodotti proponendo soluzioni a pacchetto o personalizzate. 
La società opera principalmente a livello nazionale con sedi in Veneto e Lombardia.

\paragraph*{}Per la realizzazione delle applicazioni, lo strumento che viene adottato maggiormente è InDe. Oltre allo sviluppo di software, l’azienda si occupa anche di data warehousing e business intelligence attraverso proincipalmente la suite dei prodotti Microsoft e Qlik 



\subsection{Prodotti e servizi}
\label{cap1:Prodotti e servizi}
I principali prodotti realizzati sono software e gestione di processi aziendali e di supporto al data warehousing. Essi rappresentano l'insieme dei software che supportano l'automatizzazione dei processi aziendali e di arricchimento dei dati. Si dividono principalmente nei seguenti macro gruppi:
\begin{itemize}
	\item Software di contabilità;
	\item Software per il magazzino;
	\item Software per la produzione;
	\item Software per il budgeting;
	\item Software di gestione ed analisi finanziaria;
	\item Software dedicato;
	\item Software di arricchimento dati.
\end{itemize}

\begin{figure}[!h] 
	\centering 
	\includegraphics[width=0.8\columnwidth]{immagineServizi} 
	\caption{Prodotti e servizi forniti nella loro pagina web}
\end{figure}

\azienda\ persegue due diverse soluzioni per la creazione dei prodotti software: a pacchetto e su misura. Le soluzioni a pacchetto consistono in software completi già disponibili all'interno dell'azienda destinati alla vendita. Tuttavia, la loro vendita non è immediata ma segue comunque un controllo e modifica per adattare il prodotto venduto alla realtà del cliente. L'altro tipo di soluzione consiste, invece, nella realizzazione da zero di un prodotto. Questo tipo di servizio prevede tutti i passaggi dallo studio del problema alla realizzazione del prodotto completo.

Infine, per quanto riguarda i servizi, l'azienda fornisce la manutenzione di un qualsiasi prodotto InDe, sia esso creato da \azienda\ o da una qualsiasi altra ditta che faccia affidamento a tale piattaforma, purché si disponga del codice sorgente. Inoltre, tra i servizi offerti troviamo anche: formazione su InDe, affiancamento ai progetti con InDe, conversione dei file Excel ed Access in applicazioni web e la possibilità di sfruttare prodotti web per i processi aziendali.


\subsection{Tecnologie di riferimento}
\label{cap1:Tecnologie di riferimento}
\subsubsection*{Linguaggi di programmazione}

L'azienda opera nell'ambito web. I prodotti realizzati si basano sui seguenti linguaggi di programmazione lato server: 
\begin{itemize}
	\item \textbf{C\#}: linguaggio di programmazione orientato agli oggetti che consente di creare una vasta gamma di applicazioni protette e affidabili per .NET Framework. Esso può essere adottato per creare applicazioni client Windows, servizi Web XML, componenti distribuiti, applicazioni client\-server, applicazioni di database e molto altro.
	
	\item \textbf{Java}: linguaggio di programmazione ad alto livello, orientato agli oggetti e a tipizzazione statica, che si appoggia sull'omonima piattaforma software, specificamente progettato per essere il più possibile indipendente dalla piattaforma hardware di esecuzione.
\end{itemize}

Per quanto riguarda la componente grafica, le tecnologie adottate sono: HTML5, CSS3 e Javascript.

\begin{figure}[!h] 
	\centering 
	\includegraphics[width=0.8\columnwidth]{Tecnologie} 
	\caption{Linguaggi di programmazione}
\end{figure}

\subsubsection*{Database}
Tutte le applicazioni dell'azienda mirano alla realizzazione di software gestionale, i quali necessitano di uno o più database, o per realtà più grandi dei Data Warehouse. I principali  database che hanno avuto modo di incontrare nello svolgimento delle loro attività sono stati: MySql, DB2, PostgreSQL, Oracle e SQL Server.\\

Tra i differenti database disponibili, quello adottato \azienda è principalmente SQL Server. La scelta ricade su questo dispositivo perché rappresenta un punto di incontro tra prestazioni, flessibilità e costi.\\
Altri fattori degni di nota riguardano il suo elevato utilizzo da parte delle aziende del territorio e per la sua popolarità visto che ancora oggi risulta essere il terzo database più usato al mondo dopo Oracle e MySQL. 

\begin{figure}[!h] 
	\centering 
	\includegraphics[width=0.8\columnwidth]{database} 
	\caption{Database}
\end{figure}


\section{Processi aziendali}
\label{cap1:Processi aziendali}

Questa sezione presenta l'organizzazione dell'azienda e come quest'ultima cerchi di migliorarsi nel corso del tempo. 

\subsection{Miglioramento della qualità dei processi}
\label{cap1:Miglioramento della qualità dei processi}

\azienda nello svolgimento delle proprie attività opta per perseguire il costante miglioramento dei processi. Per fare questo fa affidamento al ciclo PDCA che si compone di quattro attività:
\begin{itemize}
	\item \textbf{P}lan: individuazione degli obiettivi di miglioramento e creazione di un piano d'azione nello svolgimento dei lavori;
	\item \textbf{D}o: esecuzione di quanto pianificato;
	\item \textbf{C}heck: analisi dei risultati ottenuti nella fase precedente e confronto con quanto pianificato; 
	\item \textbf{A}ct: standardizzazione delle attività andate a buon fine e rielaborazione di quelle da migliorare ricominciando con la pianificazione.
\end{itemize}

\begin{figure}[!h] 
	\centering 
	\includegraphics[width=0.8\columnwidth]{PDCA} 
	\caption{Ciclo PDCA}
\end{figure}

Durante lo stage, ho notato che l'azienda adotta questa strategia principalmente nel processo di sviluppo mirando ad ottenere un prodotto efficiente ed efficacie. Negli altri processi aziendali invece, quali ad esempio la documentazione, spesso viene volontariamente scelto di dargli un importanza marginale. Questa decisione è legato al software che in alcuni casi si presta alla prototipizzazione rapida (\hyperref[cap1:Metodologia agile]{Sezione \ref{cap1:Metodologia agile}}), discussione con il cliente e successiva revisione. 
 
% La documentazione realizzata non è sempre completa. Viene preferito affidarsi ai mock\-up ed ai documenti che descrivono le caratteristiche tecniche e grafiche del prodotto in maniera poco completa, quando l'ideale sarebbe indagare su questi punti, ma in maniera più dettagliata fin dall'inizio. Infatti, nel corso del progetto mi sono trovato diverse volte a chiedere informazioni al tutor aziendale (responsabile di progetto), in merito a delle funzionalità non espresse nei documenti redatti e a me consegnati.  

\subsection{Metodologia agile}
\label{cap1:Metodologia agile}

L'azienda nello sviluppo delle applicazioni adotta la metodologia agile. \\
Questo metodo operativo permette una maggiore libertà rispetto ad altri tipi quali sequenziale, incrementale o a spirale. I punti fondamentali sono:
\begin{itemize}
	\item privilegiare la realizzazione del software alla creazione di documentazione;
	\item collaborare con il cliente invece di dedicarsi a contrattazioni;
	\item essere pronti a reagire ad ogni situazione invece di avere un piano di gestione dei rischi.
\end{itemize}
\azienda\ ha deciso di adottare questo metodo lavorativo per i suoi prodotti perché hanno osservato come nella realtà lavorativa le aziende vorrebbero avere a disposizione prodotti efficienti ed efficaci in tempi molto brevi. 
Inoltre, la scelta è ricaduta su questa tipologia per un motivo molto importante: conoscere i clienti, il mercato e creare un rapporto duraturo di fiducia. 

Questo metodo si concretizza con degli incontri la cui scadenza può essere, settimanale, bisettimanale o mensili, con i clienti.\\
Durante gli incontri si raccolgono task, migliorie da apportare ai progetti o addirittura ci si ferma dal cliente per realizzare nuove funzionalità e chiedere informazioni in maniera immediata. Così facendo la comunicazione è rapida, le mail sono ridotte ed è molto più facile comprendere le necessità delle aziende cliente osservandola dall'intero.


\section{Strumenti a supporto dei processi}
\label{cap1:Strumenti a supporto dei processi}
Questa sezione illustra gli strumenti adoperati a supporto dei processi e per lo sviluppo mirati a garantire qualità dei prodotti e servizi.

\subsection{Gestione di progetto}
\label{cap1:Gestione di progetto}

La gestione di progetto consiste nel definire ed organizzare il lavoro da svolgere in tempi e modi ben definiti. Per il seguente processo vengono utilizzati tre strumenti: Microsoft Teams, iDo e le mail.

\paragraph{Micorsoft Teams} è un'applicazione di comunicazione unificata multi-piattaforma . Essa permette di creare diversi gruppi con all'interno molti canali di comunicazione ai quali possono accedere solo le persone invitate. L'azienda crea un gruppo per ogni cliente e all'interno prevede un canale generale dove inserire documentazione o fare domande di natura generica. Mentre gli altri riguardano un progetto specifico o le singole funzionalità da implementare, se vi è un unico progetto. 

\begin{figure}[!h] 
	\centering 
	\includegraphics[height=0.4\columnwidth]{Teams} 
	\caption{Conversazione su Teams}
\end{figure}


\paragraph{Mail} ovvero la posta elettronica. Per iniziare il progetto mi è stata fornita una mail aziendale. Le mail servono per comunicare in maniera tempestiva la creazione ed assegnazione di una commessa nell'applicazione iDo. Queste ultime sono il principale mezzo di comunicazione con le aziende clienti. Una prassi interna all'azienda prevede che per informazioni da chiedere al cliente, bisogna prima discuterne internamente tra i membri del gruppo assegnato a quel progetto e poi quella di mettere in copia carbone il responsabile di progetto alla eventuale mail da inviare ai clienti.



\paragraph{iDo} è un'applicazione web realizzata con InDe dove vengono assegnate le commesse, indicando tempi di inizio e fine previsti. In questo applicativo, si devono indicare le ore svolte dai lavoratori specificando le ore di inizio, fine e informazioni di quanto si è svolto in quel periodo. Inoltre, si possono inserire commenti utili all'azienda cliente e a \azienda. 
Grazie a questa applicazione viene calcolato il compenso e il consuntivo del progetto. Essa si compone di diverse sezioni la Kanban (\hyperref[ido]{figura \ref{ido}}), dove vengono presentate tutte le commesse con in testa il nome del cliente e del progetto; una sezione Commessa, nella quale vengono inseriti i progetti assegnati e navigando al suo interno si accede alle commesse; una sezione tempi, dedicata a sistemare eventuali errori di inserimento a fine mese o per controllare le attività svolte dai dipendenti e prendere le opportune decisioni.

\begin{figure}[!h] 
	\centering 
	\includegraphics[width=0.8\columnwidth]{iDo} 
	\caption{Kanban dell'applicazione iDo (con nome cliente censurato)}
	\label{ido}
\end{figure}



\subsection{Documentazione}
\label{cap1:Documentazione}

L'azienda, sebbene abbia deciso di adottare un modello agile, non è priva di documenti. Quando deve realizzare un progetto il primo passo è quello di redigere un Piano di progetto e POC documentale con le principali caratteristiche che il prodotto finale dovrà avere. 
Il motivo per cui la società dà molta importanza al POC è che, con un documento nel quale è riportato la struttura del database e la grafica pressapochista che il progetto dovrà avere, permette di realizzare un prodotto completo in 3/4 settimane. 
La documentazione è salvata interamente su OneDrive For Business. Ogni documento ha una sua collocazione da rispettare. 

\subsection{Sistema di versionamento}
\label{cap1:Sistema di versionamento}

Per il versionamento e il salvataggio dei File prodotti durante la realizzazione dei progetti è previsto l'utilizzo di una repository creata dal Project Manager su un'applicazione web ideato sempre su InDe, TeamWorks. Successivamente, vengono forniti ai dipendenti scelti nella realizzazione di uno specifico progetto i permessi di: scrittura, lettura ed eliminazione.
Questa applicazione web risulta essere molto simile a GitHub, tuttavia è molto più intuitivo e semplice perché le funzionalità permesse sono controllare informazioni relative ai commit, tornare indietro di versione (rollback), scaricare progetti (Download) e creare dei derivati (Fork) premendo unicamente dei pulsanti.

Ciascun progetto deve essere soggetto a versionamento perciò chiunque lo utilizza ha una visione chiara e dettagliata della sua storia e delle sue modifiche. Ad ogni task deve corrispondere una versione. Nelle versioni viene applicato il seguente formalismo:
\begin{center}
	X.Y.Z
\end{center}
Dove X,Y,Z sono numeri incrementali da 0 a infinito. 
Z indica i singoli task e bug individuati ed risolti, Y rappresenta la implementazione di nuove funzionalità rilasciate per la fase di collaudo e X quando il progetto ha superato il collaudo e diventa operativo.  

\subsubsection{Sistema di pubblicazione}
La pubblicazione presso \azienda\ corrisponde al rilascio dell'applicazione al cliente per effettuare dei test. Il collaudo viene effettuato anche internamente all'azienda, ma questa doppio controllo permette di realizzare applicazioni corrette che non necessitino di eccessive manutenzioni di tipo correttivo. 
Il sistema adottato per pubblicare il software è IDManager, anche essa una applicazione web di InDe, la quale permette di modificare i riferimenti del database cambiando la stringa di connessione del progetto e permette di caricare unicamente le differenze tra la versione precedente e quella attuale.


\subsection{Ambiente di sviluppo}
\label{cap1:Ambiente di sviluppo}

\paragraph{\inde} consiste in una piattaforma ad alta produttività, per lo sviluppo di applicazioni cross-platform (Web-based) creata da Pro Gamma S.p.A.. La scelta dell'azienda ricade su questo tipo di strumento per i seguenti motivi:
\begin{itemize}
	\item Scrivere l'applicazione e poterla distribuire in ambiente Java o Microsoft C\#;
	\item Collegare ed utilizzare più database contemporaneamente anche di tipo differente;
	\item Implementare applicazioni Desktop e Mobile;
	\item Gestire i rilasci successivi in maniera sicura e strutturata;
	\item Potersi focalizzare sui processi da gestire, sui i dati da memorizzare o modificare, evitando di dover programmare a basso livello, avendo però la possibilità, quando necessario, di poterlo fare.
\end{itemize}
Le applicazioni prodotte sono perciò nativamente multi-piattaforma, cross-browser, multi-database già nel momento in cui vengono create.

\paragraph{Microsoft SQL Server} Un DBMS relazionale, prodotto da Microsoft, che usa T-SQL, una variante del linguaggio SQL Standard. 

\paragraph{Microsoft SQL Server Management Studio} \'E un'applicazione software che viene utilizzata per la configurazione, la gestione e l'amministrazione di tutti i componenti all'interno di Microsoft SQL Server. Lo strumento include sia editor di script che strumenti grafici che funzionano con oggetti e funzionalità del server.

\paragraph{Qlik}\'E un pacchetto che include QlikView, Qlik Sense ed NPrinting. Questi software sono di visualizzazione e business intelligence che permettono il rapido sviluppo di dashboard completamente personalizzabili in grado di fornire rapidamente informazioni utili sui dati a disposizione.

\paragraph{Microsoft Power BI}
\'E un servizio di analisi aziendale di Microsoft fortemente integrato con Microsoft Office e con gli altri strumenti dell'ecosistema Microsoft. Mira a fornire visualizzazioni interattive e funzionalità di business intelligence con un'interfaccia abbastanza semplice in modo da consentire agli utenti finali di creare i propri report e dashboard.

\subsubsection*{Altri strumenti}
Oltre agli strumenti appena descritti, eventuali IDE per scrivere in C\#, Java e gli altri linguaggi riportati in Sezione \ref{cap1:Tecnologie di riferimento}, sono lasciati al programmatore. Può capitare che nel corso di un progetto siano richieste  modifiche specifiche che l'applicazione InDe non permette, in quei casi vi è una modalità di inserimento personalizzato che consente di scrivere codice.



\subsection{Sistemi operativi}
\label{cap1:Sistemi operativi}

L'azienda usa  solo i sistemi operativi di Windows. Questo perché risultano essere gli unici compatibili con InDe. Per chi non dovesse avere a disposizione tale sistema operativo viene fornita una macchina virtuale alla quale collegarsi. 

\subsubsection*{VPN e desktop remoto}
In base al progetto, spesso può capitare che ci si debba affidare alla macchina virtuale del cliente. In queste occasioni l'azienda consiglia l'utilizzo di una delle seguenti applicazioni per connettersi alla VPN: openVPN, FortiClient o lo strumento di Windows. 
Mentre per entrare in desktop remoto: Connessione Desktop Remoto di windows oppure nRemoteNG, il quale offre anche la possibilità di creare una o più connessione VPN ed aprire diversi schermi remoti contemporaneamente. 
Quando invece si effettuano delle assistenze, AnyDesk o TeamViewer sono dei software efficienti per collegarsi al desktop del cliente e risolvere problemi.


\section{Clientela}
\label{cap1:Clientela}
I clienti di \azienda\ risiedono nei territori del nord Italia. La sede legale dell'azienda si trova a Milano. A Castelfranco veneto è collocata una delle due sedi operative, presso la quale ho svolto la mia esperienza di stage.\\
I clienti sono imprese di medio-grandi dimensioni. Altre imprese degne di nota sono Aton s.p.a, Sistemi s.p.a, WiseEnergy Italia s.r.l, Geox s.p.a. ed altre aziende che operano a livello internazionale.

\paragraph*{}In seguito alla compilazione dell'\hyperref[NDA]{Accordo di non divulgazione} per l'intera durata del progetto non verrà nominato il nome dell'azienda cliente.

\newpage