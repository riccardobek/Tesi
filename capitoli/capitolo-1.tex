% !TEX encoding = UTF-8
% !TEX TS-program = pdflatex
% !TEX root = ../tesi.tex


\chapter{L'Azienda}
\label{cap:introduzione}
%**************************************************************

%Introduzione al contesto applicativo.\\

%\noindent Esempio di utilizzo di un termine nel glossario \\
%\gls{api}. \\

%\noindent Esempio di citazione in linea \\
%\cite{site:agile-manifesto}. \\

%\noindent Esempio di citazione nel pie' di pagina \\
%citazione\footcite{womak:lean-thinking} \\

%**************************************************************
\section{Presentazione azienda}
Questo capitolo ...

\subsection{\azienda}

\azienda è una software house specializzata nello sviluppo di applicazioni gestionali attraverso l'utilizzo di strumenti CASE. Nasce ... negli anni ci sono stati ... inizialmente ... L'azienda opera principalmente a livello nazionale con sedi in Veneto e Lombardia. Essa propone soluzioni a pacchetto e/o realizzate da specifiche richieste del cliente.

Per la realizzazione delle applicazioni, lo strumento che viene adottato maggiormente è \inde\ (InDe), una piattaforma ad alta produttività, per lo sviluppo di applicazioni cross-platform (Web-based) creata da Pro Gamma S.p.A..
La scelta dell'azienda ricade su questo tipo di strumento per i seguenti vantaggi:
\begin{itemize}
	\item Scrivere l'applicazione e poterla distribuire in ambiente Java o Microsoft C\#;
	\item Collegare ed utilizzare più database contemporaneamente anche di tipo differente;
	\item Implementare applicazioni Desktop e Mobile;
	\item Gestire i rilasci successivi in maniera sicura e strutturata;
	\item Potersi focalizzare sui processi da gestire, sui i dati da memorizzare o modificare, evitando di dover programmare a basso livello, avendo però la possibilità, quando necessario, di poterlo fare.
\end{itemize}
Le applicazioni prodotte sono perciò nativamente multipiattaforma, crossbrowser, multidatabase già nel momento in cui vengono create.
Oltre allo sviluppo di software l'azienda si occupa, attraverso i partener strategici, anche di Data Warehousing e Business Intelligence principalmente attraverso la suite dei prodotti Qlik e Microsoft Power BI.\\

\subsection{Prodotti e servizi}

\subsection{Tecnologie di riferimento}


\section{Processi aziendali}
\subsection{Miglioramento della qualità dei processi}
\subsection{Metodologia agile}

\section{Strumenti a supporto dei processi}
\subsection{Gestione di progetto}
Parlare di teams, mail, iDo

\subsection{Documentazione}
parlare dei documenti che ho dovuto redigere e dei documenti redatti per prendere accordi con i diversi clienti.

\subsection{Sistema di versionamento}
parlare delle funionalità di teamworks e di IDManager

\subsection{Ambiente di sviluppo}
Parlare di Instant developer e microsoft sql server.
Oltre agli strumenti appena descritti, eventuali IDE per scrivere in C\#, Java, CSS sono lasciati al programmatore. Può capitare che nel corso di un progetto siano richieste  modifiche specifiche che l'applicazione \inde\ non permette, in quei casi l'applicazione permette di scrivere codice proprio indicando la cartella dove sono contenute le modifiche specifiche. (Riportare immagine \todo).

\subsection{Sistemi operativi}
Parlare della libera scelta del sistema operativo da adottare. In alcuni casi è richiesto l'uso delle macchine virtuali adottando preferibilmente uno tra i segeunti struemtni: quello di microsoft o mRemoteNG che permette di stare connesso a diverse macchine virtuali contemporaneamente. Infine parlare delle VPN che sono necessarie


\section{Clientela}
