\chapter{Valutazione retrospettiva}
\section{\inde}

\subsection{\inde: creazione di una schermata base}
In questa sezione vorrei analizzare il particolare il contributo di Inde al mio progetto. Illustro come una pagina web viene creata da zero per permettere ai lettori di questo documento di comprendere quanto può essere semplice, con l'aiuto di strumenti RAD come questo, creare applicazioni.

\subsubsection{Passo 1: Database}
La prima attività da svolgere è quella di pensare al proprio database, se è necessario. Quest'ultima affermazione è legata al fatto che è possibile creare applicazioni anche senza un database.
In questo contesto ideiamo un semplice database per la gestione delle corsie di un supermercato. Le tabelle che creiamo sono Corsia e Articoli.

Per creare le due tabelle è necessario conoscere basi di dati soprattutto per evitare creare fin da subito query non troppo complicate che faticano ad essere completate. 

Con il pulsante destro clicco due volte sul database definisco le specifiche che lo riguardano. A questo punto premo il pulsante destro del mouse sul database e seleziono la voce "Aggiungi tabella". Con lo stesso procedimento di creazione tabella, creo i campi.
Se desidero creare delle foreign key si deve trascinare la tabella interessata verso quella di destinazione.  

\todo immagine e ulteriori spiegazioni

\subsubsection{Passo 2: Oggetto}
Una volta creata la tabella, per creare un oggetto è sufficiente trascinare sull'applicazione la tabella tenendo premuto shift e ctrl. Viene generato un documento (classe) che potremo gestire come meglio crediamo. In questo caso mi limito a creare l'oggetto.


\todo immagine e ulteriori spiegazioni

\subsubsection{Passo 3: Videata}
Trascinando l'oggetto sull'applicazione e premendo o shift o ctrl viene creata una videata basata sull'oggetto. Poi selezionando la videata interessata la si può modificare graficamente e gestire le funzionalità.


\todo immagine e ulteriori spiegazioni

\subsection{Estensioni}
In questo contesto non sono è risultato necessario adoperare questa funzionalità. Tuttavia, ritengo importante accennare alle possibilità offerte dall'applicazione. InDe, infatti, permette di estendere le sue librerie con delle nuove. Ho visto che è possibile creare delle funzioni in SQL e anche implementare esternamente un file Java o C\# e richiamarlo nell'applicazione. 
L'aspetto negativo delle estensioni è che una volta implementate in Java o C\# o nel linguaggio del database (Oracle, SQL Server, MySQL) non si può cambiare tecnologia, salvo il caso in cui non si siano implementate le estensioni per tutti i linguaggi.

\todo immagine e ulteriori spiegazioni

\section{Obiettivi}
In questa sezione indico il raggiungimento degli obiettivi pianificati con l'azienda e quelli che mi sono prefissato di raggiungere al termine dello stage.

\subsection{Stage}
Lo stage si è svolto in 310 ore esattamente come da pianificazione. Le attività hanno avuto tempi diversi da quanto previsto nel Piano di lavoro: l'apprendimento del software, delle Stored Procedures e dei formalismi hanno richiesto quattro giorni, permettendomi di anticipare lo studio dell'applicazione, ad oggi, realizzata.  L'implementazione e rilascio del progetto ha richiesto la totalità del tempo lavorativo, non potendo affrontare l'aspetto delle stored procedures. 
La seguente tabella illustra lo stato di soddisfacimento dei requisti:

\begin{center}
	\begin{tabular}{ l|l|c }
		\hline
		
		\multicolumn{2}{c}{\textbf{Obbligatori}} & \textbf{Soddisfatto} \\
		\hline
		\textit{O01} & Apprendimento della piattaforma Instant Developer & Si\\
		\textit{O02} & Test delle funzionalità implementate e rilascio & Si\\
		\textit{O03} & Utilizzo di Microsoft SQL Server& Si\\
		\hline
		\multicolumn{2}{c}{\textbf{Desiderabili}} & \textbf{Soddisfatto}\\
		\hline
		\textit{D01} & Gestione di progetto & Si\\
		\textit{D02} & Comunicazione con il cliente & Si\\
		\textit{D03} & Scrittura delle procedure T-SQL& No\\
		\hline
		\multicolumn{2}{c}{\textbf{Facoltativi}} & \textbf{Soddisfatto}\\
		\hline
		\textit{F01} & Autonomia a risolvere nuove problematiche & Si
		
	\end{tabular}
\end{center}
	

\subsection{Personali}
Per quanto riguarda gli obiettivi personali, mi ritengo pienamente soddisfatto. Ho potuto apprendere un tipo di programmazione che mira molto sulla logica e meno sul codice. Inoltre, ho potuto osservare dei database di grandi dimensioni non sempre gestiti nella maniera ottimale e comprenderne le funcionalità e lescelte che portano la realizzazione di alcune delle tabelle. Infine, ho imparato quanto difficile può essere la comunicazione tra cliente e fornitore.
Riassumendo, gli obiettivi personali raggiunti ma che ritengo possano ulteriormente migliorare sono:


\begin{center}
	\begin{tabular}{l|l}
		\hline
		\textbf{Personali}& \textbf{Soddisfatto} \\
		\hline
		Accrescere le conoscenze in merito al mondo RAD e Data Warehouse & Si\\
		Migliorare le capacità di realizzazione di applicazioni seguendo il metodo Bottom-Up & Migliorabile\\
		Apprendere come interfacciarmi con i clienti & Si\\
		Migliorare le mie capacità di Problem Solving & Si

	\end{tabular}
\end{center}


\section{Considerazioni personali}
